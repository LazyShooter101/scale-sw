% Copyright (C) 2017 Daniel Page <dan@phoo.org>
%
% Use of this source code is restricted per the CC BY-SA license, a copy of
% which can be found via http://creativecommons.org (and should be included 
% as LICENSE.txt within the associated archive or repository).

\documentclass[crop={false},multi={true},tikz={true}]{standalone}

\ifstandalone
\usepackage{libbuild-style}
\usepackage{libbuild-macro}
\fi

\ifstandalone
\addbibresource{libbuild.bib}
\fi

\begin{document}

% -----------------------------------------------------------------------------

\subsection{Background}

Imagine you encounter a server, denoted $\PARTY{D}$, 
which houses an ageing but still operational $32$-bit Katmai model Intel 
Pentium III processor.  
A datasheet you find for this processor lists it having a $16$ kilobyte, 
$4$-way set-associative L1 data cache with $32$ byte cache lines.  Since
around $1999$, the server has performed a mission-critical back-office 
role in a banking application: you discover it acts as a primitive form 
of HSM, performing DES encryption on behalf of front-end clients.
By leveraging access to the network, 
an attacker $\PARTY{E}$ can interact with $\PARTY{D}$ as follows:

\begin{center}
% Copyright (C) 2017 Daniel Page <dan@phoo.org>
%
% Use of this source code is restricted per the CC BY-SA license, a copy of
% which can be found via http://creativecommons.org (and should be included 
% as LICENSE.txt within the associated archive or repository).

\begin{tikzpicture}

% =============================================================================

% configure
 \tikzset{node distance={6cm},party/.style={rectangle,minimum width={2.0cm},minimum height={2.0cm}}}

% adversary and target device
 \node [                party] (E) {$\PARTY{E}$} ;
 \node [right of=E,draw,party] (D) {$\PARTY{D}$} ;

% embedded material
 \node at (D.south east) [anchor=south east,draw,rectangle] {\tiny $\PRI{\CHAL{k}}$} ;

%   direct input and output
 \draw [       >=stealth,->] (E.30)    to              
                             node [above]             {$\PUB{P}$} 
                             (E.30    -| D.west)  ;
 \draw [       >=stealth,<-] (E.330)   to              
                             node [below]             {$\PUB{Q}$} 
                             (E.330   -| D.west)  ;

% indirect input and output
%\draw [dashed,>=stealth,<-] (D.north) to [bend right] 
%                            node [above]             {} 
%                            (D.north -| E.north) ;
%\draw [dashed,>=stealth,->] (D.south) to [bend left]  
%                            node [below]             {} 
%                            (D.south -| E.south) ;

% computation
 \draw [       >=stealth,->] (D.30)    to [bend left,in=90,out=90,looseness=2] 
                             node [right,anchor=west] {$\PRI{Q} = \PMUL{\PRI{\CHAL{k}}}{\PUB{P}}$}
                             (D.330) ;

% =============================================================================

\end{tikzpicture}

\end{center}

\noindent
That is, in each interaction $\PARTY{E}$ can (adaptively) send 
a chosen DES  plaintext
to $\PARTY{D}$; the device will
encrypt that  plaintext under the fixed, unknown DES cipher key $\PRI{\CHAL{k}}$,
and produce 
the corresponding ciphertext.
In addition, $\PARTY{E}$ can measure the time $\PARTY{D}$ takes to execute 
each operation: it approximates this 
(keeping in mind there may be some experimental noise) 
by simply timing how long $\PARTY{D}$ takes to respond with $\PUB{c}$ once 
provided with $\PUB{m}$.

\subsection{Materials}

\subsubsection{\lstinline[language={bash}]|\$\{ARCHIVE\}/CONF(ARCHIVE_PATH,CID)/\$\{USER\}.D|}

This executable simulates the attack target $\PARTY{D}$.  When executed it 
reads the following input

\begin{itemize}
\item $\PUB{m}$, 
      a  ${1}$-block DES plaintext 
      (represented as an           octet string)
\end{itemize}

\noindent
from \lstinline[language={bash}]{stdin} and writes the following output

\begin{itemize}
\item $\LEAK$,
      an execution time measured in clock cycles
      (represented as a  decimal integer string),
      and
\item $\PUB{c}$,
      a  ${1}$-block DES ciphertext
      (represented as an           octet string)
\end{itemize}

\noindent
to \lstinline[language={bash}]{stdout}, in both cases with one field per 
line.  Execution continues this way, i.e., by repeatedly reading input 
then writing output, until it is forcibly terminated (or crashes).  
Note that:

\begin{itemize}
\item $\PARTY{D}$ uses a software implementation of DES, based on that
      written by Richard Outerbridge~\cite[Part V]{SCALE:Schneier:06}.
\item DES has a $64$-bit block length, and, although it notionally accepts 
      a $64$-bit cipher key, the {\em effective} cipher key length is in
      fact $56$ bits: the remaining $8$ bits support error detection via 
      a parity code.  You can ignore the parity bits, meaning {\em many}
      $64$-bit candidate cipher keys will be valid when used in place of 
      $\PRI{\CHAL{k}}$ (since only the $56$ bits actually used by DES are 
      relevant).
\item The cited source code uses a set of eight separate S-box look-up 
      tables.  Each look-up table has $64$ entries of type 
      \lstinline[language=C]{unsigned long} 
      (i.e., a $64$-bit unsigned integer), so will consume $512$ bytes 
      of memory.  Doing so improves efficiency: a look-up table captures 
      the function of an associated S-box {\em plus} the (inefficient) 
      P- and E-permutations.
      However, the implementation {\em used} differs slightly different.  
      Given each of the look-up table entries holds $32$ bits of content, 
      the type of entries has been altered to 
      \lstinline[language=C]{unsigned int} 
      Once loaded, an entry is simply cast into 
      \lstinline[language=C]{unsigned long},
      roughly halving the memory footprint with no impact on efficiency.
      Given the $32$-byte L1 cache line size, and that each look-up table 
      is aligned to a $32$-byte boundary, each cache line can therefore 
      accommodate eight full $32$-bit entries.
\item To perform the role described, the implementation

      \begin{enumerate}
      \item invokes the \lstinline[language=C]{deskey} function
            to pre-compute the round keys from a given cipher key,
            then
      \item invokes the \lstinline[language=C]{des}    function
            to perform an encryption operation.
      \end{enumerate}

      \noindent
      Within the 
      \lstinline[language=C]{des}
      function, 
      \lstinline[language=C]{scrunch} and \lstinline[language=C]{unscrun}
      are used to convert a sequence of eight $8$-bit bytes to and from a
      $64$-bit integer (held as two $32$-bit halves) before and after 
      \lstinline[language=C]{desfunc}
      performs the encryption operation itself.
\item You can assume that when an encryption operation is performed, i.e.,
      when the
      \lstinline[language=C]{des}
      function is invoked, the cache contains {\em no} S-box content: any
      cache-hits or -misses during encryption therefore depend only on 
      $\PRI{\CHAL{k}}$ and $\PUB{m}$, as does the execution time.
\end{itemize}

\subsubsection{\lstinline[language={bash}]|\$\{ARCHIVE\}/CONF(ARCHIVE_PATH,CID)/\$\{USER\}.R|}

In many side-channel and fault attacks, the attacker is able to perform an 
initial profiling or calibration phase: a common rationale for doing so is 
to support selection or fine-tuning of parameters for a subsequent attack.  
With this in mind, a simulated replica $\PARTY{R}$ of the attack target is 
also provided.
$\PARTY{R}$ is identical to $\PARTY{D}$, bar accepting input of the form

\begin{itemize}
\item $\PUB{\REPL{m}}$, 
      a $1$-block DES plaintext
      (represented as an           octet string),
      and
\item $\PUB{\REPL{k}}$, 
      a           DES cipher key
      (represented as an           octet string)
\end{itemize}

\noindent
on \lstinline[language={bash}]{stdin} (again with one field per line):
in contrast to $\PARTY{D}$, this means $\PARTY{R}$ uses the
{\em chosen} DES cipher key $\PUB{\REPL{k}}$
to encrypt the
     chosen  DES  plaintext $\PUB{\REPL{m}}$.

\subsection{Tasks}

\begin{enumerate}
\item Write a program that simulates the adversary $\PARTY{E}$ by attacking
      the simulated target, or, more specifically, that recovers the target 
      material $\PRI{\CHAL{k}}$.  
      When executed using a command of the form

      \begin{lstlisting}[language={bash},gobble={6}]
      bash$ ./attack ${USER}.D
      \end{lstlisting}

      \noindent
      the attack should be invoked on the simulated target named (not some
      hard-coded alternative).  Use \lstinline[language={bash}]{stdout} to 
      print 
      a) any intermediate output you deem relevant, followed finally by 
      b) two lines which clearly detail the target material recovered plus
         the total number of interactions with attack target.
\item Answer the exam-style questions in 
      \lstinline[language={bash}]|${ARCHIVE}/${USER}/CONF(ARCHIVE_PATH,CID)/${USER}.exam|.
\end{enumerate}

% -----------------------------------------------------------------------------

\ifstandalone
\printbibliography
\fi

\end{document}
