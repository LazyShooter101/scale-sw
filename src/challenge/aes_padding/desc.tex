% Copyright (C) 2017 Daniel Page <dan@phoo.org>
%
% Use of this source code is restricted per the CC BY-SA license, a copy of
% which can be found via http://creativecommons.org (and should be included 
% as LICENSE.txt within the associated archive or repository).

\documentclass[crop={false},multi={true},tikz={true}]{standalone}

\ifstandalone
\usepackage{libbuild-style}
\usepackage{libbuild-macro}
\fi

\ifstandalone
\addbibresource{libbuild.bib}
\fi

\begin{document}

% -----------------------------------------------------------------------------

\subsection{Background}

Imagine you encounter a server, denoted $\PARTY{D}$, 
which acts as a storage repository for sensor data.  An installation of $50$ 
or so IoT-class sensor nodes regularly transmit encrypted packets of data to
$\PARTY{D}$, which stores them for later analysis (iff. they are deemed to be
valid).  Having already captured some encrypted network traffic transmitted 
from a particular sensor node to the server, you tasked with recovering the 
underlying data.
By leveraging access to the network, 
an attacker $\PARTY{E}$ can interact with $\PARTY{D}$ as follows:

\begin{center}
% Copyright (C) 2017 Daniel Page <dan@phoo.org>
%
% Use of this source code is restricted per the CC BY-SA license, a copy of
% which can be found via http://creativecommons.org (and should be included 
% as LICENSE.txt within the associated archive or repository).

\begin{tikzpicture}

% =============================================================================

% configure
 \tikzset{node distance={6cm},party/.style={rectangle,minimum width={2.0cm},minimum height={2.0cm}}}

% adversary and target device
 \node [                party] (E) {$\PARTY{E}$} ;
 \node [right of=E,draw,party] (D) {$\PARTY{D}$} ;

% embedded material
 \node at (D.south east) [anchor=south east,draw,rectangle] {\tiny $\PRI{\CHAL{k}}$} ;

%   direct input and output
 \draw [       >=stealth,->] (E.30)    to              
                             node [above]             {$\PUB{P}$} 
                             (E.30    -| D.west)  ;
 \draw [       >=stealth,<-] (E.330)   to              
                             node [below]             {$\PUB{Q}$} 
                             (E.330   -| D.west)  ;

% indirect input and output
%\draw [dashed,>=stealth,<-] (D.north) to [bend right] 
%                            node [above]             {} 
%                            (D.north -| E.north) ;
%\draw [dashed,>=stealth,->] (D.south) to [bend left]  
%                            node [below]             {} 
%                            (D.south -| E.south) ;

% computation
 \draw [       >=stealth,->] (D.30)    to [bend left,in=90,out=90,looseness=2] 
                             node [right,anchor=west] {$\PRI{Q} = \PMUL{\PRI{\CHAL{k}}}{\PUB{P}}$}
                             (D.330) ;

% =============================================================================

\end{tikzpicture}

\end{center}

\noindent
That is, in each interaction $\PARTY{E}$ can (adaptively) send 
a chosen AES ciphertext
to $\PARTY{D}$; the device

\begin{enumerate}
\item decrypts $\PUB{c}$, i.e., computes
      \[
      \PRI{m} \CONS \PRI{\TAG} \CONS \PRI{\PAD} = \SCOPE{\ID{AES-CBC}}{\ALG{Dec}}( \PRI{\CHAL{k}_1}, \PUB{iv}, \PUB{c} )
      \]
      then
\item checks whether $\PRI{\PAD}$ is valid, 
      aborting immediately  and producing result code $1$ if not,
      then
\item checks whether $\PRI{\TAG}$ is valid, i.e., whether 
      \[
      \SCOPE{\ID{HMAC-SHA-1}}{\ALG{Ver}}( \PRI{\CHAL{k}_2}, \PRI{m}, \PRI{\TAG} ) = \TRUE ,
      \]
      aborting immediately  and producing result code $2$ if not,
      then
\item processes $\PRI{m}$
                            and produces  result code $0$.
\end{enumerate}

\noindent
Note that the associated  plaintext is {\em not} produced explicitly as
output.

\subsection{Materials}

\subsubsection{\lstinline[language={bash}]|\$\{ARCHIVE\}/\$\{USER\}/CONF(ARCHIVE_PATH,CID)/\$\{USER\}.D|}

This executable simulates the attack target $\PARTY{D}$.  When executed it 
reads the following input

\begin{itemize}
\item $\PUB{l}$,
      a  length
      (represented as a  decimal integer string),
\item $\PUB{iv}$,
      a      ${1}$-block AES-CBC initialisation vector
      (represented as an octet string),
      and
\item $\PUB{c}$,
      an $\PUB{l}$-block AES-CBC ciphertext
      (represented as an           octet string)
\end{itemize}

\noindent
from \lstinline[language={bash}]{stdin} and writes the following output

\begin{itemize}
\item $\LEAK$,
      a  result code
      (represented as a  decimal integer string)
\end{itemize}

\noindent
to \lstinline[language={bash}]{stdout}, in both cases with one field per 
line.  Execution continues this way, i.e., by repeatedly reading input 
then writing output, until it is forcibly terminated (or crashes).  
Note that:

\begin{itemize}
\item To avoid timing attacks, $\PARTY{D}$ makes use of a high-performance
      yet constant-time AES-128 implementation (operated in CBC mode) that
      is derived from techniques in~\cite{SCALE:KasSch:09};
      this clearly implies $128$-bit block and cipher key lengths.  
\item Keep in mind some limits, namely $0 \leq \PUB{l} < 256$, on maximum
      plaintext and/or ciphertext length.
\item The padding scheme used matches that of TLS.  This means
      \[
      \PRI{\PAD} = \LIST{ \underbrace{ \alpha, \alpha, \ldots, \alpha }_{\mbox{$\alpha$ + 1 octets}} }
      \]
      is a sequence of $\alpha + 1$ octets, each of whose value is $\alpha$
      and where $0 \leq \alpha < 256$, constructed st. the AES-CBC input is
      always a multiple of the block size.
\end{itemize}

\subsubsection{\lstinline[language={bash}]|\$\{ARCHIVE\}/\$\{USER\}/CONF(ARCHIVE_PATH,CID)/\$\{USER\}.conf|}

This file represents a set of attack parameters, with everything (e.g.,
all public values) $\PARTY{E}$ has access to by default.  It contains 

\begin{itemize}
\item $\PUB{\CHAL{iv}}$,
      a  ${1}$-block AES-CBC initialisation vector
      (represented as an           octet string),
      and
\item $\PUB{\CHAL{c}}$,
      a  ${1}$-block AES-CBC ciphertext 
      (represented as an           octet string),
      corresponding to an encryption of some unknown plaintext 
      $\PRI{\CHAL{m}}$ (using $\PUB{\CHAL{iv}}$)
\end{itemize}

\noindent
with one field per-line.
More specifically, this represents the previously captured network traffic
whose decryption, i.e., recovery of $\PRI{\CHAL{m}}$, is the task at hand.
\IfStrEqCase{CONF(CHALLENGE,CID)}{%
  {0}{Keep in mind that $\PRI{\CHAL{m}}$ will 
      include the SHA-1 hash          of \lstinline[language={bash}]{$\{USER\}}
      as the least-significant octets: 
      this allows candidate decryptions to be checked for validity.
  }%
  {1}{Keep in mind that $\PRI{\CHAL{m}}$ will 
      include an ASCII representation of \lstinline[language={bash}]{$\{USER\}} 
      as the least-significant octets: 
      this allows candidate decryptions to be checked for validity.
  }%
  {2}{Keep in mind that $\PRI{\CHAL{m}}$ will have been
      be generated entirely at random:
      this means checking validity of candidate decryptions is more 
      difficult than it would be otherwise.
  }%
}%

\subsection{Tasks}

\begin{enumerate}
\item Write a program that simulates the adversary $\PARTY{E}$ by attacking
      the simulated target, or, more specifically, that recovers the target 
      material $\PRI{\CHAL{m}}$.  
      When executed using a command of the form

      \begin{lstlisting}[language={bash},gobble={6}]
      bash$ ./attack ${USER}.D ${USER}.conf
      \end{lstlisting}

      \noindent
      the attack should be invoked on the simulated target named (not some
      hard-coded alternative).  Use \lstinline[language={bash}]{stdout} to 
      print 
      a) any intermediate output you deem relevant, followed finally by 
      b) two lines which clearly detail the target material recovered plus
         the total number of interactions with attack target.
\item Answer the exam-style questions in 
      \lstinline[language={bash}]|${ARCHIVE}/${USER}/CONF(ARCHIVE_PATH,CID)/${USER}.exam|.
\end{enumerate}

% -----------------------------------------------------------------------------

\ifstandalone
\printbibliography
\fi

\end{document}
