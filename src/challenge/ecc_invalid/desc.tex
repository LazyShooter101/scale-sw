% Copyright (C) 2017 Daniel Page <dan@phoo.org>
%
% Use of this source code is restricted per the CC BY-SA license, a copy of
% which can be found via http://creativecommons.org (and should be included 
% as LICENSE.txt within the associated archive or repository).

\documentclass[crop={false},multi={true},tikz={true}]{standalone}

\ifstandalone
\usepackage{libbuild-style}
\usepackage{libbuild-macro}
\fi

\ifstandalone
\addbibresource{libbuild.bib}
\fi

\begin{document}

% -----------------------------------------------------------------------------

\subsection{Background}

Imagine you own a cryptographic accelerator, denoted $\PARTY{D}$: it is used
as a low-cost HSM in your data center, supporting secure key generation and 
storage plus off-load of various cryptographic operations via a standardised 
protocol (or API).  $\PARTY{D}$ is primarily used to perform elliptic curve
scalar multiplication, meaning you can interact with it as follows:

\begin{center}
% Copyright (C) 2017 Daniel Page <dan@phoo.org>
%
% Use of this source code is restricted per the CC BY-SA license, a copy of
% which can be found via http://creativecommons.org (and should be included 
% as LICENSE.txt within the associated archive or repository).

\begin{tikzpicture}

% =============================================================================

% configure
 \tikzset{node distance={6cm},party/.style={rectangle,minimum width={2.0cm},minimum height={2.0cm}}}

% adversary and target device
 \node [                party] (E) {$\PARTY{E}$} ;
 \node [right of=E,draw,party] (D) {$\PARTY{D}$} ;

% embedded material
 \node at (D.south east) [anchor=south east,draw,rectangle] {\tiny $\PRI{\CHAL{k}}$} ;

%   direct input and output
 \draw [       >=stealth,->] (E.30)    to              
                             node [above]             {$\PUB{P}$} 
                             (E.30    -| D.west)  ;
 \draw [       >=stealth,<-] (E.330)   to              
                             node [below]             {$\PUB{Q}$} 
                             (E.330   -| D.west)  ;

% indirect input and output
%\draw [dashed,>=stealth,<-] (D.north) to [bend right] 
%                            node [above]             {} 
%                            (D.north -| E.north) ;
%\draw [dashed,>=stealth,->] (D.south) to [bend left]  
%                            node [below]             {} 
%                            (D.south -| E.south) ;

% computation
 \draw [       >=stealth,->] (D.30)    to [bend left,in=90,out=90,looseness=2] 
                             node [right,anchor=west] {$\PRI{Q} = \PMUL{\PRI{\CHAL{k}}}{\PUB{P}}$}
                             (D.330) ;

% =============================================================================

\end{tikzpicture}

\end{center}

\noindent
That is, in each interaction you can (adaptively) send 
a point
to $\PARTY{D}$; the accelerator will
multiply that point by the fixed, unknown scalar $\PRI{\CHAL{k}}$,
and produce 
the corresponding point.
Although it currently functions correctly, an electrical fault demands that
$\PARTY{D}$ is replaced.  However, doing so presents a serious problem: the 
API cannot be used to export $\PRI{\CHAL{k}}$, so a replacement will have 
to be generated and used thereafter.  This means compatibility with various
existing data sets will be lost, {\em unless} $\PRI{\CHAL{k}}$ itself can 
be recovered by some other means.

\subsection{Material}

\subsubsection{\lstinline[language={bash}]|\$\{ARCHIVE\}/CONF(ARCHIVE_PATH,CID)/\$\{USER\}.D|}

This executable simulates the attack target $\PARTY{D}$.  When executed it 
reads the following input

\begin{itemize}
\item $\PUB{P_x}$,
      the $x$-coordinate of elliptic curve point $\PUB{P} = \TUPLE{ \PUB{P_x}, \PUB{P_y} }$
      (represented as a  hexadecimal integer string),      
      and
\item $\PUB{P_y}$,
      the $y$-coordinate of elliptic curve point $\PUB{P} = \TUPLE{ \PUB{P_x}, \PUB{P_y} }$
      (represented as a  hexadecimal integer string)
\end{itemize}

\noindent
from \lstinline[language={bash}]{stdin} and writes the following output

\begin{itemize}
\item $\PUB{Q_x}$,
      the $x$-coordinate of elliptic curve point $\PUB{Q} = \TUPLE{ \PUB{Q_x}, \PUB{Q_y} }$
      (represented as a  hexadecimal integer string),      
      and
\item $\PUB{Q_y}$,
      the $y$-coordinate of elliptic curve point $\PUB{Q} = \TUPLE{ \PUB{Q_x}, \PUB{Q_y} }$
      (represented as a  hexadecimal integer string)
\end{itemize}

\noindent
to \lstinline[language={bash}]{stdout}, in both cases with one field per 
line.  Execution continues this way, i.e., by repeatedly reading input 
then writing output, until it is forcibly terminated (or crashes).  
Note that:

\begin{itemize}
\item $\PARTY{D}$ uses an affine, left-to-right binary scalar multiplication
      (i.e., an additive analogue of~\cite[Section 2.1]{SCALE:Gordon:85}) to 
      compute $\PUB{Q} = \PMUL{\PRI{\CHAL{k}}}{\PUB{P}}$: it assumes use of 
      the
      \IfStrEqCase{CONF(CURVE,CID)}{%
        {0}{NIST-P-192 }% NIST-P-192
        {1}{NIST-P-224 }% NIST-P-224
        {2}{NIST-P-256 }% NIST-P-256
        {3}{NIST-P-384 }% NIST-P-384
        {4}{NIST-P-521 }% NIST-P-521
      } elliptic curve
      \[
      E( \DOM{\B{F}_p} ) : y^2 = x^3 + \DOM{a_4} x + \DOM{a_6}
      \]
      for standardised~\cite{SCALE:FIPS:186:00} domain parameters $\DOM{p}$, 
      $\DOM{a_4}$ and $\DOM{a_6}$.
\item More concretely, $\PARTY{D}$ uses OpenSSL to support arithmetic on the
      elliptic curve.  You can browse the associated source code at
      \[
      \mbox{\url{http://github.com/openssl/}}
      \]
      noting that the choice outlined above implies use of 
      \lstinline[language={C}]{ec_GFp_simple_add}
      and
      \lstinline[language={C}]{ec_GFp_simple_dbl}
      to realise the (additive) group operation, i.e., to compute either a
      point addition or doubling, within the scalar multiplication.
\item Crucially, $\PARTY{D}$ applies {\em no} checks to either the input or
      output points (e.g., to check whether 
      $
      \PUB{P} \in E( \DOM{\B{F}_p} )
      $ 
      and reject $\PUB{P}$ if not).
\end{itemize}

\subsection{Tasks}

\begin{enumerate}
\item Write a program that simulates the adversary $\PARTY{E}$ by attacking
      the simulated target, or, more specifically, that recovers the target 
      material $\PRI{\CHAL{k}}$.  
      When executed using a command of the form
      \[
      \mbox{\lstinline[language={bash}]|./attack \$\{USER\}.D|}
      \]
      the attack should be invoked on the simulated target named (not some
      hard-coded alternative).  Use \lstinline[language={bash}]{stdout} to 
      print 
      a) any intermediate output you deem relevant, followed finally by 
      b) two lines which clearly detail the target material recovered plus
         the total number of interactions with attack target.
\item Answer the exam-style questions in 
      \lstinline[language={bash}]|${ARCHIVE}/CONF(ARCHIVE_PATH,CID)/${USER}.exam|.
\end{enumerate}

% -----------------------------------------------------------------------------

\ifstandalone
\printbibliography
\fi

\end{document}
