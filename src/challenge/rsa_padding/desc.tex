% Copyright (C) 2017 Daniel Page <dan@phoo.org>
%
% Use of this source code is restricted per the CC BY-SA license, a copy of
% which can be found via http://creativecommons.org (and should be included 
% as LICENSE.txt within the associated archive or repository).

\documentclass[crop={false},multi={true},tikz={true}]{standalone}

\ifstandalone
\usepackage{libbuild-style}
\usepackage{libbuild-macro}
\fi

\ifstandalone
\addbibresource{libbuild.bib}
\fi

\begin{document}

% =============================================================================

\ifstandalone
\author{\url{http://www.github.com/danpage/scale}}
\title{\Large SCALE challenge description : \lstinline[language={bash}]|\$\{ARCHIVE\}/CONF(ARCHIVE_PATH,CID)|}
\date{\today}

\maketitle
\fi

% =============================================================================

\leveldown{Background}

Imagine you form part of a red team\footnote{
\url{https://en.wikipedia.org/wiki/Red_team}
} asked to assess a specific e-commerce server, denoted $\PARTY{D}$, 
which houses a $64$-bit Intel Core2 processor.  
In reality, the server is a HSM-like device representing the back-office 
infrastructure that supports various secure web-sites: the server offers
secure key generation and storage, plus off-load for some cryptographic 
operations.  In particular, it is able to compute RSAES-OAEP decryptions, 
as standardised by PKCS\#1 v2.1~\cite{SCALE:RFC:3447}, by using an 
embedded RSA private key.
By leveraging access to the network, 
an attacker $\PARTY{E}$ can interact with $\PARTY{D}$ as follows:

\begin{center}
% Copyright (C) 2017 Daniel Page <dan@phoo.org>
%
% Use of this source code is restricted per the CC BY-SA license, a copy of
% which can be found via http://creativecommons.org (and should be included 
% as LICENSE.txt within the associated archive or repository).

\begin{tikzpicture}

% =============================================================================

% configure
 \tikzset{node distance={6cm},party/.style={rectangle,minimum width={2.0cm},minimum height={2.0cm}}}

% adversary and target device
 \node [                party] (E) {$\PARTY{E}$} ;
 \node [right of=E,draw,party] (D) {$\PARTY{D}$} ;

% embedded material
 \node at (D.south east) [anchor=south east,draw,rectangle] {\tiny $\PRI{\CHAL{k}}$} ;

%   direct input and output
 \draw [       >=stealth,->] (E.30)    to              
                             node [above]             {$\PUB{P}$} 
                             (E.30    -| D.west)  ;
 \draw [       >=stealth,<-] (E.330)   to              
                             node [below]             {$\PUB{Q}$} 
                             (E.330   -| D.west)  ;

% indirect input and output
%\draw [dashed,>=stealth,<-] (D.north) to [bend right] 
%                            node [above]             {} 
%                            (D.north -| E.north) ;
%\draw [dashed,>=stealth,->] (D.south) to [bend left]  
%                            node [below]             {} 
%                            (D.south -| E.south) ;

% computation
 \draw [       >=stealth,->] (D.30)    to [bend left,in=90,out=90,looseness=2] 
                             node [right,anchor=west] {$\PRI{Q} = \PMUL{\PRI{\CHAL{k}}}{\PUB{P}}$}
                             (D.330) ;

% =============================================================================

\end{tikzpicture}

\end{center}

\noindent
That is, in each interaction $\PARTY{E}$ can (adaptively) send 
a chosen RSAES-OAEP label and ciphertext
to $\PARTY{D}$; the device will
decrypt that ciphertext under the fixed, unknown RSA private key $\TUPLE{ \PUB{\CHAL{N}}, \PRI{\CHAL{d}} }$ 
and produce 
a result code based on validity of the underlying  plaintext.
Note that the associated  plaintext is {\em not} produced explicitly as
output.

RSAES-OAEP decryption makes use of Optimal Asymmetric Encryption Padding 
(OAEP)~\cite{SCALE:BelRog:94}, per~\cite[Section 7.1]{SCALE:RFC:3447}.  
Crucially, the decryption process can encounter various error conditions:

\begin{description}
\item[Error \NUM{1}:]
     A decryption error occurs in Step $3.g$ of \ALG{RSAES-OAEP-Decrypt} 
     if the octet string passed to the decoding phase
     does {\em not} have a most-significant $\RADIX{00}{16}$ octet.
     Put another way, the error occurs because 
     the output produced by RSA decryption is too large to fit into one 
     fewer octets than the modulus.

\item[Error \NUM{2}:]
     A decryption error occurs in Step $3.g$ of \ALG{RSAES-OAEP-Decrypt} 
     if the octet string passed into the decoding phase 
     does {\em not} a) produce a hashed label that matches, or b) use a
     $\RADIX{01}{16}$ octet between any padding and the message.
     Put another way, the error occurs because 
     the plaintext validity checking mechanism fails.
\end{description}

\noindent
A footnote~\cite[Section 7.1.2]{SCALE:RFC:3447} explains why all errors, 
these two in particular, should be indistinguishable from each other (that
is, a given application should not reveal {\em which} error occurred, only 
that {\em some} error occurred).  The implementation used by $\PARTY{D}$ 
does {\em not} adhere to this advice, meaning the error is exposed via 
the result code produced.

% -----------------------------------------------------------------------------

\levelstay{Material}

\leveldown{\lstinline[language={bash}]|\$\{ARCHIVE\}/CONF(ARCHIVE_PATH,CID)/\$\{USER\}.D|}

This executable simulates the attack target $\PARTY{D}$.  When executed it 
reads the following input

\begin{itemize}
\item $\PUB{l}$,
      an RSAES-OAEP label
      (represented as a  length-prefixed, hexadecimal octet   string),
      and
\item $\PUB{c}$,
      an RSAES-OAEP ciphertext
      (represented as a  length-prefixed, hexadecimal octet   string),
\end{itemize}

\noindent
from \lstinline[language={bash}]{stdin} and writes the following output

\begin{itemize}
\item $\LEAK$,
      a  result code
      (represented as a                       decimal integer string),
\end{itemize}

\noindent
to \lstinline[language={bash}]{stdout}, in both cases with one field per 
line.  Execution continues this way, i.e., by repeatedly reading input 
then writing output, until it is forcibly terminated (or crashes).  
The result code should be interpreted as follows:

\begin{itemize}
\item If the decryption was a success                   
      then the result code is $0$.
\item If error \NUM{1} occurred during decryption 
      then the result code is $1$.
\item If error \NUM{2} occurred during decryption 
      then the result code is $2$.
\item If there was some other internal error (e.g., due to malformed input)
      then the result code {\em attempts} to tell you why: 

      \begin{itemize}
      \item If the result code is $3$ then \ALG{RSAEP}              
            failed because the operand was out of range 
            (section $5.1.1$, step $1$, page $11$), 
            i.e., the  plaintext is not between $0$ and $\PUB{\CHAL{N}}-1$.
      \item If the result code is $4$ then \ALG{RSADP}              
            failed because the operand was out of range 
            (section $5.1.2$, step $1$, page $11$), 
            i.e., the ciphertext is not between $0$ and $\PUB{\CHAL{N}}-1$.
      \item If the result code is $5$ then \ALG{RSAES-OAEP-Encrypt} 
            failed because a length check failed        
            (section $7.1.1$, step $1.b$, page $18$), 
            i.e., the message is too long.
      \item If the result code is $6$ then \ALG{RSAES-OAEP-Decrypt} 
            failed because a length check failed        
            (section $7.1.2$, step $1.b$, page $20$), 
            i.e., the ciphertext does not match the length of $\PUB{\CHAL{N}}$.
      \item If the result code is $7$ then \ALG{RSAES-OAEP-Decrypt} 
            failed because a length check failed        
            (section $7.1.2$, step $1.c$, page $20$), 
            i.e., the ciphertext does not match the length of the hash function output.
      \end{itemize}

      \noindent
      Any other result code (of $8$ upward) implies an abnormal error whose
      cause cannot be directly associated with any of the above.
\end{itemize}

\levelstay{\lstinline[language={bash}]|\$\{ARCHIVE\}/CONF(ARCHIVE_PATH,CID)/\$\{USER\}.conf|}

This file represents a set of attack parameters, with everything (e.g.,
all public values) $\PARTY{E}$ has access to by default.  It contains 

\begin{itemize}
\item $\PUB{\CHAL{N}}$,     
      an RSA        modulus
      (represented as a                   hexadecimal integer string),
\item $\PUB{\CHAL{e}}$,
      an RSA        public exponent
      (represented as a                   hexadecimal integer string),
      st. $\PUB{\CHAL{e}} \cdot \PRI{\CHAL{d}} \equiv 1 \pmod{\PRI{\Phi{(\CHAL{N})}}}$,
\item $\PUB{\CHAL{l}}$,
      an RSAES-OAEP label
      (represented as a  length-prefixed, hexadecimal octet   string),
      and
\item $\PUB{\CHAL{c}}$,
      an RSAES-OAEP ciphertext 
      (represented as a  length-prefixed, hexadecimal octet   string),
      corresponding to an encryption of some unknown plaintext 
      $\PRI{\CHAL{m}}$ (using $\PUB{\CHAL{l}}$),
\end{itemize}

\noindent
with one field per line.
More specifically, this represents the RSA public key 
$
\TUPLE{ \PUB{\CHAL{N}}, \PUB{\CHAL{e}} }
$
associated with the unknown RSA private key 
$
\TUPLE{ \PUB{\CHAL{N}}, \PRI{\CHAL{d}} } 
$
embedded in $\PARTY{D}$, 
plus a ciphertext $\PUB{\CHAL{c}}$ whose decryption, i.e., the recovery 
of $\PRI{\CHAL{m}}$, is the task at hand.  You can assume the RSAES-OAEP 
encryption of $\PRI{\CHAL{m}}$ (that produced $\PUB{\CHAL{c}}$) used the
\ALG{MGF1} mask generation function with 
\IfStrEqCase{CONF(HASH,CID)}{%
  {0}  {SHA-1 }% SHA-1
  {1}{SHA-256 }% SHA-256
  {2}{SHA-384 }% SHA-384
  {3}{SHA-512 }% SHA-512
} as the underlying hash function.
\IfStrEqCase{CONF(CHALLENGE,CID)}{%
  {0}{Additionally, 
      $\PRI{\CHAL{m}}$ will 
      include the SHA-1 hash          of \lstinline[language={bash}]{$\{USER\}}
      as the least-significant octets: 
      this allows candidate decryptions to be checked for validity.
  }%
  {1}{Additionally, 
      $\PRI{\CHAL{m}}$ will 
      include an ASCII representation of \lstinline[language={bash}]{$\{USER\}} 
      as the least-significant octets: 
      this allows candidate decryptions to be checked for validity.
  }%
  {2}{Additionally, 
      $\PRI{\CHAL{m}}$ will have been
      be generated entirely at random:
      this means checking validity of candidate decryptions is more 
      difficult than it would be otherwise.
  }%
}%

% -----------------------------------------------------------------------------

\levelup{Tasks}

\begin{enumerate}
\item Write a program that simulates the adversary $\PARTY{E}$ by attacking
      the simulated target, or, more specifically, that recovers the target 
      material $\PRI{\CHAL{m}}$.  
      When executed using a command of the form
      \[
      \mbox{\lstinline[language={bash}]|./attack \$\{USER\}.D \$\{USER\}.conf|}
      \]
      the attack should be invoked on the simulated target named (not some
      hard-coded alternative).  Use \lstinline[language={bash}]{stdout} to 
      print 
      a) any intermediate output you deem relevant, followed finally by 
      b) two lines which clearly detail the target material recovered plus
         the total number of interactions with the attack target.
\item Answer the exam-style questions in 
      \lstinline[language={bash}]|${ARCHIVE}/CONF(ARCHIVE_PATH,CID)/${USER}.exam|.
\end{enumerate}

% =============================================================================

\ifstandalone
\printbibliography
\fi

% =============================================================================

\ifstandalone
\appendix 
\section{Representation and conversion}

\subsection{Integers}

\paragraph{Concept.}
An integer string (or literal) is written as a string of characters, 
each of which represents a digit; the set of possible digits, and the 
value being represented, depends on a base $b$.  We read digits from 
right-to-left: the least-significant (resp. most-significant) digit 
is the right-most (resp. left-most) character within the string.
As such, the $20$-character string
\[
      \REP{x}  =        \mbox{\lstinline|09080706050403020100|}
\]
represents the integer value
\[
      \begin{array}{lcl}
      \REP{x} &\mapsto& \begin{array}{*{5}{l@{\;}c@{\;}lc}}
                        \REP{x}_{19}  &\cdot&  b^{19} &+&
                        \REP{x}_{18}  &\cdot&  b^{18} &+&
                        \REP{x}_{17}  &\cdot&  b^{17} &+&
                        \REP{x}_{16}  &\cdot&  b^{16} &+&
                        \REP{x}_{15}  &\cdot&  b^{15} &+\\
                        \REP{x}_{14}  &\cdot&  b^{14} &+&
                        \REP{x}_{13}  &\cdot&  b^{13} &+&
                        \REP{x}_{12}  &\cdot&  b^{12} &+&
                        \REP{x}_{11}  &\cdot&  b^{11} &+&
                        \REP{x}_{10}  &\cdot&  b^{10} &+\\
                        \REP{x}_{ 9}  &\cdot&  b^{ 9} &+&
                        \REP{x}_{ 8}  &\cdot&  b^{ 8} &+&
                        \REP{x}_{ 7}  &\cdot&  b^{ 7} &+&
                        \REP{x}_{ 6}  &\cdot&  b^{ 6} &+&
                        \REP{x}_{ 5}  &\cdot&  b^{ 5} &+\\                    
                        \REP{x}_{ 4}  &\cdot&  b^{ 4} &+&
                        \REP{x}_{ 3}  &\cdot&  b^{ 3} &+&
                        \REP{x}_{ 2}  &\cdot&  b^{ 2} &+&
                        \REP{x}_{ 1}  &\cdot&  b^{ 1} &+&
                        \REP{x}_{ 0}  &\cdot&  b^{ 0} & \\
                        \end{array}
      \end{array}
\]
Of course the actual value depends on the base $b$ in which we interpret
the representation $\REP{x}$:

\begin{itemize}
\item for a     decimal integer string $b=10$, meaning
      \[
      \begin{array}{lcl}
      \REP{x} &\mapsto& \begin{array}{*{5}{l@{\;}c@{\;}lc}}
                        \REP{x}_{19}  &\cdot& 10^{19} &+&
                        \REP{x}_{18}  &\cdot& 10^{18} &+&
                        \REP{x}_{17}  &\cdot& 10^{17} &+&
                        \REP{x}_{16}  &\cdot& 10^{16} &+&
                        \REP{x}_{15}  &\cdot& 10^{15} &+\\
                        \REP{x}_{14}  &\cdot& 10^{14} &+&
                        \REP{x}_{13}  &\cdot& 10^{13} &+&
                        \REP{x}_{12}  &\cdot& 10^{12} &+&
                        \REP{x}_{11}  &\cdot& 10^{11} &+&
                        \REP{x}_{10}  &\cdot& 10^{10} &+\\
                        \REP{x}_{ 9}  &\cdot& 10^{ 9} &+&
                        \REP{x}_{ 8}  &\cdot& 10^{ 8} &+&
                        \REP{x}_{ 7}  &\cdot& 10^{ 7} &+&
                        \REP{x}_{ 6}  &\cdot& 10^{ 6} &+&
                        \REP{x}_{ 5}  &\cdot& 10^{ 5} &+\\
                        \REP{x}_{ 4}  &\cdot& 10^{ 4} &+&
                        \REP{x}_{ 3}  &\cdot& 10^{ 3} &+&
                        \REP{x}_{ 2}  &\cdot& 10^{ 2} &+&
                        \REP{x}_{ 1}  &\cdot& 10^{ 1} &+&
                        \REP{x}_{ 0}  &\cdot& 10^{ 0} & \\
                        \end{array}                     \\\\
              &\mapsto& \begin{array}{*{5}{l@{\;}c@{\;}lc}}
                        \RADIX{0}{10} &\cdot& 10^{19} &+&
                        \RADIX{9}{10} &\cdot& 10^{18} &+&
                        \RADIX{0}{10} &\cdot& 10^{17} &+&
                        \RADIX{8}{10} &\cdot& 10^{16} &+&
                        \RADIX{0}{10} &\cdot& 10^{15} &+\\
                        \RADIX{7}{10} &\cdot& 10^{14} &+&
                        \RADIX{0}{10} &\cdot& 10^{13} &+&
                        \RADIX{6}{10} &\cdot& 10^{12} &+&
                        \RADIX{0}{10} &\cdot& 10^{11} &+&
                        \RADIX{5}{10} &\cdot& 10^{10} &+\\
                        \RADIX{0}{10} &\cdot& 10^{ 9} &+&
                        \RADIX{4}{10} &\cdot& 10^{ 8} &+&
                        \RADIX{0}{10} &\cdot& 10^{ 7} &+&
                        \RADIX{3}{10} &\cdot& 10^{ 6} &+&
                        \RADIX{0}{10} &\cdot& 10^{ 5} &+\\
                        \RADIX{2}{10} &\cdot& 10^{ 4} &+&
                        \RADIX{0}{10} &\cdot& 10^{ 3} &+&
                        \RADIX{1}{10} &\cdot& 10^{ 2} &+&
                        \RADIX{0}{10} &\cdot& 10^{ 1} &+&
                        \RADIX{0}{10} &\cdot& 10^{ 0} & \\
                        \end{array}                     \\\\
              &\mapsto& \RADIX{9080706050403020100}{10}
      \end{array}
      \]
      whereas
\item for a hexadecimal integer string $b=16$, meaning
      \[
      \begin{array}{lcl}
      \REP{x} &\mapsto& \begin{array}{*{5}{l@{\;}c@{\;}lc}}
                        \REP{x}_{19}  &\cdot& 16^{19} &+&
                        \REP{x}_{18}  &\cdot& 16^{18} &+&
                        \REP{x}_{17}  &\cdot& 16^{17} &+&
                        \REP{x}_{16}  &\cdot& 16^{16} &+&
                        \REP{x}_{15}  &\cdot& 16^{15} &+\\
                        \REP{x}_{14}  &\cdot& 16^{14} &+&
                        \REP{x}_{13}  &\cdot& 16^{13} &+&
                        \REP{x}_{12}  &\cdot& 16^{12} &+&
                        \REP{x}_{11}  &\cdot& 16^{11} &+&
                        \REP{x}_{10}  &\cdot& 16^{10} &+\\
                        \REP{x}_{ 9}  &\cdot& 16^{ 9} &+&
                        \REP{x}_{ 8}  &\cdot& 16^{ 8} &+&
                        \REP{x}_{ 7}  &\cdot& 16^{ 7} &+&
                        \REP{x}_{ 6}  &\cdot& 16^{ 6} &+&
                        \REP{x}_{ 5}  &\cdot& 16^{ 5} &+\\
                        \REP{x}_{ 4}  &\cdot& 16^{ 4} &+&
                        \REP{x}_{ 3}  &\cdot& 16^{ 3} &+&
                        \REP{x}_{ 2}  &\cdot& 16^{ 2} &+&
                        \REP{x}_{ 1}  &\cdot& 16^{ 1} &+&
                        \REP{x}_{ 0}  &\cdot& 16^{ 0} & \\
                        \end{array}                     \\\\
              &\mapsto& \begin{array}{*{5}{l@{\;}c@{\;}lc}}
                        \RADIX{0}{16} &\cdot& 16^{19} &+&
                        \RADIX{9}{16} &\cdot& 16^{18} &+&
                        \RADIX{0}{16} &\cdot& 16^{17} &+&
                        \RADIX{8}{16} &\cdot& 16^{16} &+&
                        \RADIX{0}{16} &\cdot& 16^{15} &+\\
                        \RADIX{7}{16} &\cdot& 16^{14} &+&
                        \RADIX{0}{16} &\cdot& 16^{13} &+&
                        \RADIX{6}{16} &\cdot& 16^{12} &+&
                        \RADIX{0}{16} &\cdot& 16^{11} &+&
                        \RADIX{5}{16} &\cdot& 16^{10} &+\\
                        \RADIX{0}{16} &\cdot& 16^{ 9} &+&
                        \RADIX{4}{16} &\cdot& 16^{ 8} &+&
                        \RADIX{0}{16} &\cdot& 16^{ 7} &+&
                        \RADIX{3}{16} &\cdot& 16^{ 6} &+&
                        \RADIX{0}{16} &\cdot& 16^{ 5} &+\\
                        \RADIX{2}{16} &\cdot& 16^{ 4} &+&
                        \RADIX{0}{16} &\cdot& 16^{ 3} &+&
                        \RADIX{1}{16} &\cdot& 16^{ 2} &+&
                        \RADIX{0}{16} &\cdot& 16^{ 1} &+&
                        \RADIX{0}{16} &\cdot& 16^{ 0} & \\
                        \end{array}                     \\\\
              &\mapsto& \RADIX{42649378395939397566720}{10}
      \end{array}
      \]
\end{itemize}

\paragraph{Example.}
Consider the following Python $3.x$ program
 
\begin{lstlisting}[language={Python}]
t_0 = '09080706050403020100'

t_1 = int( t_0, 10 )
t_2 = int( t_0, 16 )

t_3 = ( '{0:d}'.format( t_2 ) )
t_4 = ( '{0:X}'.format( t_2 ) )

t_5 = ( '{0:X}'.format( t_2 ) ).zfill( 20 )

print( 'type( t_0 ) = {0!s:14s} t_0 = {1!s}'.format( type( t_0 ), t_0 ) )
print( 'type( t_1 ) = {0!s:14s} t_1 = {1!s}'.format( type( t_1 ), t_1 ) )
print( 'type( t_2 ) = {0!s:14s} t_2 = {1!s}'.format( type( t_2 ), t_2 ) )
print( 'type( t_3 ) = {0!s:14s} t_3 = {1!s}'.format( type( t_3 ), t_3 ) )
print( 'type( t_4 ) = {0!s:14s} t_4 = {1!s}'.format( type( t_4 ), t_4 ) )
print( 'type( t_5 ) = {0!s:14s} t_5 = {1!s}'.format( type( t_5 ), t_5 ) )
\end{lstlisting}
 
\noindent
which, when executed, produces
 
\begin{lstlisting}[language={bash}]
type( t_0 ) = <class 'str'>  t_0 = 09080706050403020100
type( t_1 ) = <class 'int'>  t_1 = 9080706050403020100
type( t_2 ) = <class 'int'>  t_2 = 42649378395939397566720
type( t_3 ) = <class 'str'>  t_3 = 42649378395939397566720
type( t_4 ) = <class 'str'>  t_4 = 9080706050403020100
type( t_5 ) = <class 'str'>  t_5 = 09080706050403020100
\end{lstlisting}

\noindent
as output.  This is intended to illustrate that
 
\begin{itemize}
\item \lstinline[language={Python}]|t_0| is an
      integer string
      (i.e., a sequence of characters),
\item \lstinline[language={Python}]|t_1| 
      and 
      \lstinline[language={Python}]|t_2|
      are conversions of 
      \lstinline[language={Python}]|t_0| 
      into
      integers,
      using decimal and hexadecimal respectively,
      and
\item \lstinline[language={Python}]|t_3| 
      and
      \lstinline[language={Python}]|t_4|
      are conversions of 
      \lstinline[language={Python}]|t_2| 
      into
      strings
      (i.e., a sequence of characters),
      using decimal and hexadecimal respectively.
\end{itemize}
 
\noindent
Note that 
\lstinline[language={Python}]|t_0| 
and 
\lstinline[language={Python}]|t_4| 
do not match: 
the conversion ignored the left-most zero character, because it is not
significant wrt. the associated integer value.  
If/when this issue is problematic, it can be resolved by using the
\lstinline[language={Python}]|zfill| 
function to left-fill the string (with zero characters) so it is of the 
required length.
\lstinline[language={Python}]|t_0| 
and
\lstinline[language={Python}]|t_5| 
do     match,
for example,
because the latter has been filled to ensure an overall length of $20$ 
characters.


\subsection{Octet strings}

\paragraph{Concept.}
The term octet\footnote{
\url{http://en.wikipedia.org/wiki/Octet_(computing)}
} is normally used as a synonym for byte, most often within the context
of communication (and computer networks).  Using octet is arguably more 
precise than byte, in that the former is {\em always} $8$ bits whereas
the latter {\em can}\footnote{
\url{http://en.wikipedia.org/wiki/Byte}, for example, details the fact
that the term ``byte'' can be and has been interpreted to mean
a) a group of $n$ bits for $n < w$ (i.e., smaller than the word size),
b) the data type used to represent characters,
   or
c) the (smallest) unit of addressable data in memory:
although POSIX mandates $8$-bit bytes, for example, each of these cases
permits an alternative definition.  
} differ.  A string is a sequence of characters, and so, by analogy, an
octet string\footnote{
Note the octet string terminology stems from ASN.1 encoding; see, e.g., 
\url{http://en.wikipedia.org/wiki/Abstract_Syntax_Notation_One}.
} is a sequence of octets: ignoring some corner cases, it is reasonable 
to use the term ``octet string'' as a synonym for ``byte sequence''.

To represent a given byte sequence, we use what can be formally termed
a (little-endian) length-prefixed, hexadecimal octet string.  However,
doing so requires some explanation: each element of that term relates
to a property of the representation, where we define
a) little-endian\footnote{
   \url{http://en.wikipedia.org/wiki/Endianness}
   } to mean, if read left-to-right,
   the first octet represents the     $0$-th element of the source byte sequence
   and
   the last  octet represents the $(n-1)$-st element of the source byte sequence,
b) length-prefixed\footnote{
   \url{http://en.wikipedia.org/wiki/String_(computer_science)}
   } to mean $n$, the length of the source byte sequence, is prepended 
   to the octet string as a single $8$-bit\footnote{
   Although it simplifies the challenge associated with parsing such a 
   representation, note that use of an $8$-bit length implies an upper 
   limit of $255$ elements in the associated byte sequence.
   } length or ``header'' octet,
   and
c) hexadecimal\footnote{
   \url{http://en.wikipedia.org/wiki/Hexadecimal}
   } to mean each octet is represented by using $2$ hexadecimal digits.  
   Note that, confusingly, hexadecimal digits within each pair will be
   big-endian: if read left-to-right, the most-significant is first.
For convenience, we assume the term octet string is a catch-all implying
all such properties from here on.

An example likely makes all of the above {\em much} clearer: certainly
there is nothing complex involved.  Concretely, consider a $16$-element 
byte sequence 
\[
\mbox{\lstinline[language={C}]|uint8_t x[ 16 ] = \{ 0, 1, 2, 3, 4, 5, 6, 7, 8, 9, 10, 11, 12, 13, 14, 15 \}|}
\]
defined using C.  This would be represented as
\[
\REP{x} ~=~ \mbox{\lstinline|10:000102030405060708090A0B0C0D0E0F|}
\]
using a colon to separate the length and value fields:

\begin{itemize}
\item the length (LHS of the colon) is the integer
      $
      n = \RADIX{10}{16} = \RADIX{16}{10} ,
      $
      and
\item the value  (RHS of the colon) is the byte sequence
      $
      x = \LIST{ \RADIX{00}{16}, \RADIX{01}{16}, \ldots, \RADIX{0F}{16} } = \LIST{ \RADIX{0}{10}, \RADIX{1}{10}, \ldots, \RADIX{15}{10} } \equiv \mbox{\lstinline[language={C}]|x|} .
      $
\end{itemize}

\paragraph{Example.}
Consider the following Python $3.x$ program

\begin{lstlisting}[language={Python}]
import binascii

def octetstr2bytes( x ) :
  t = x.split( ':' ) ; n = int( t[ 0 ], 16 ) ; x = binascii.a2b_hex( t[ 1 ] )

  if ( n != len( x ) ) :
    raise ValueError
  else :
    return x

def bytes2octetstr( x ) :
  return ( '{0:02X}'.format( len( x ) ) ) + ':' + ( binascii.b2a_hex( x ).decode( 'ascii' ).upper() )

def bytes2seq( x ) :
  return            [ int( t ) for t in x ]

def seq2bytes( x ) :
  return bytearray( [ int( t ) for t in x ] )

t_0 = [ 0xDE, 0xAD, 0xBE, 0xEF ]
t_1 = bytes2octetstr( seq2bytes( t_0 ) )

print( 'type( t_0 ) = {0!s:14s} t_0 = {1!s}'.format( type( t_0 ), t_0 ) )
print( 'type( t_1 ) = {0!s:14s} t_1 = {1!s}'.format( type( t_1 ), t_1 ) )

t_2 = '04:DEADBEEF'
t_3 = bytes2seq( octetstr2bytes( t_2 ) )

print( 'type( t_2 ) = {0!s:14s} t_2 = {1!s}'.format( type( t_2 ), t_2 ) )
print( 'type( t_3 ) = {0!s:14s} t_3 = {1!s}'.format( type( t_3 ), t_3 ) )
\end{lstlisting}

\noindent
which, when executed, produces

\begin{lstlisting}[language={bash}]
type( t_0 ) = <class 'list'>   t_0 = [222, 173, 190, 239]
type( t_1 ) = <class 'str'>    t_1 = 04:DEADBEEF
type( t_2 ) = <class 'str'>    t_2 = 04:DEADBEEF
type( t_3 ) = <class 'list'>   t_3 = [222, 173, 190, 239]
\end{lstlisting}

\noindent
as output.  This is intended to illustrate that

\begin{itemize}
\item If you start with a
      list (or sequence)
      \lstinline[language={Python}]|t_0|, 
      then 
      \lstinline[language={Python}]|seq2bytes| 
      and 
      \lstinline[language={Python}]|bytes2octetstr|
      will convert it first into a byte sequence then an octet string:
      the fact that
      $\mbox{\lstinline[language={Python}]|t_1|} = \mbox{\lstinline[language={Python}]|04:DEADBEEF|}$
      implies 
      $n = \RADIX{04}{16} = \RADIX{04}{10}$
      and
      $x = \LIST{ \RADIX{DE}{16}, \RADIX{AD}{16}, \RADIX{BE}{16}, \RADIX{EF}{16} }$,
      i.e., 
      $n = \mbox{\lstinline[language={Python}]|len( t_0 )|}$
      and
      $x = \mbox{\lstinline[language={Python}]|     t_0  |}$,
      so, for example,
      $x_1 = \mbox{\lstinline[language={Python}]|t_0[ 1 ]|} = \RADIX{AD}{16}$.

\item If you start with an
      octet string  
      \lstinline[language={Python}]|t_2|, 
      then 
      \lstinline[language={Python}]|octetstr2bytes| and \lstinline[language={Python}]|bytes2seq|
      will convert it first into a byte sequence then a list (or sequence):
      the fact that
      $\mbox{\lstinline[language={Python}]|t_2|} = \mbox{\lstinline[language={Python}]|04:DEADBEEF|}$
      implies 
      $n = \RADIX{04}{16} = \RADIX{04}{10}$
      and
      $x = \LIST{ \RADIX{DE}{16}, \RADIX{AD}{16}, \RADIX{BE}{16}, \RADIX{EF}{16} }$,
      i.e., 
      $n = \mbox{\lstinline[language={Python}]|len( t_3 )|}$
      and
      $x = \mbox{\lstinline[language={Python}]|     t_3  |}$,
      so, for example,
      $x_0 = \mbox{\lstinline[language={Python}]|t_3[ 0 ]|} = \RADIX{DE}{16}$.
\end{itemize}

\noindent
Note that the conversions are self-consistent in the sense that
$\mbox{\lstinline[language={Python}]|t_0|} = \mbox{\lstinline[language={Python}]|t_3|}$,
or rather 
\[
\mbox{\lstinline[language={Python}]|t_0|} = \mbox{\lstinline[language={Python}]|bytes2seq( octetstr2bytes( bytes2octetstr( seq2bytes( t_0 ) ) ) )|} .
\]

\fi

% =============================================================================

\end{document}
