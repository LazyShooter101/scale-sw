% Copyright (C) 2017 Daniel Page <dan@phoo.org>
%
% Use of this source code is restricted per the CC BY-SA license, a copy of
% which can be found via http://creativecommons.org (and should be included 
% as LICENSE.txt within the associated archive or repository).

% =============================================================================

\leveldown{Background}

Imagine you are tasked with attacking a device, denoted $\PARTY{D}$, 
which houses an $8$-bit Intel $8051$ processor.
$\PARTY{D}$ represents an ISO/IEC 7816 compliant contact-based smart-card
chip, used within larger devices as a co-processor module: it has support
for secure key generation and storage, plus off-load of some cryptographic 
operations, such as AES, via a standardised protocol (or API).
By leveraging physical access, 
an attacker $\PARTY{E}$ can interact with $\PARTY{D}$ as follows:

\begin{center}
% Copyright (C) 2017 Daniel Page <dan@phoo.org>
%
% Use of this source code is restricted per the CC BY-SA license, a copy of
% which can be found via http://creativecommons.org (and should be included 
% as LICENSE.txt within the associated archive or repository).

\begin{tikzpicture}

% =============================================================================

% configure
 \tikzset{node distance={6cm},party/.style={rectangle,minimum width={2.0cm},minimum height={2.0cm}}}

% adversary and target device
 \node [                party] (E) {$\PARTY{E}$} ;
 \node [right of=E,draw,party] (D) {$\PARTY{D}$} ;

% embedded material
 \node at (D.south east) [anchor=south east,draw,rectangle] {\tiny $\PRI{\CHAL{k}}$} ;

%   direct input and output
 \draw [       >=stealth,->] (E.30)    to              
                             node [above]             {$\PUB{P}$} 
                             (E.30    -| D.west)  ;
 \draw [       >=stealth,<-] (E.330)   to              
                             node [below]             {$\PUB{Q}$} 
                             (E.330   -| D.west)  ;

% indirect input and output
%\draw [dashed,>=stealth,<-] (D.north) to [bend right] 
%                            node [above]             {} 
%                            (D.north -| E.north) ;
%\draw [dashed,>=stealth,->] (D.south) to [bend left]  
%                            node [below]             {} 
%                            (D.south -| E.south) ;

% computation
 \draw [       >=stealth,->] (D.30)    to [bend left,in=90,out=90,looseness=2] 
                             node [right,anchor=west] {$\PRI{Q} = \PMUL{\PRI{\CHAL{k}}}{\PUB{P}}$}
                             (D.330) ;

% =============================================================================

\end{tikzpicture}

\end{center}

\noindent
That is, in each interaction $\PARTY{E}$ can (adaptively) send 
a chosen AES  plaintext
to $\PARTY{D}$; the device will
encrypt that  plaintext under the fixed, unknown AES cipher key $\PRI{\CHAL{k}}$,
and produce 
the corresponding ciphertext.
Note that $\PARTY{E}$ supplies the power and clock signals to $\PARTY{D}$.
Using an irregular clock signal (e.g., via some form of glitch to disrupt 
regular oscillation) causes $\PARTY{D}$ to malfunction.  One malfunction, 
or fault, can be induced per interaction with $\PARTY{D}$: it will act to 
randomise one element of the state matrix used by AES, at a chosen point 
during the associated encryption operation.

% -----------------------------------------------------------------------------

\levelstay{Material}

\leveldown{\lstinline[language={bash}]|\$\{ARCHIVE\}/CONF(ARCHIVE_PATH,CID)/\$\{USER\}.D|}

This executable simulates the attack target $\PARTY{D}$.  When executed it 
 reads the following  input (one field per line)

\begin{itemize}
\item $\FAULT$, 
      a  fault specification
      (whose representation is explained below),
      and
\item $\PUB{m}$,
      a  ${1}$-block AES plaintext
      (represented as a  length-prefixed, hexadecimal octet   string),
\end{itemize}

\noindent
from \lstinline[language={bash}]{stdin},
and 
writes the following output (one field per line)

\begin{itemize}
\item $\PUB{c}$,
      a  ${1}$-block AES ciphertext
      (represented as a  length-prefixed, hexadecimal octet   string),
\end{itemize}

\noindent
to   \lstinline[language={bash}]{stdout}.
Execution continues this way, i.e., repeatedly reading input then writing 
output, until $\PARTY{D}$ is forcibly terminated (or crashes).
Note that:

\begin{itemize}
\item $\PARTY{D}$ uses an AES-128 implementation,
      which clearly implies $128$-bit block and cipher key lengths;
      following the notation in~\cite[Figure 5]{SCALE:FIPS:197:01}, the 
      fact that 
      $
      Nb =  4
      $ 
      and 
      $
      Nr = 10
      $ 
      means a $( 4 \times 4 )$-element state matrix is used in a total of 
      $11$ rounds
      where

      \begin{itemize}
      \item the  $0$-th           round  consists of the
            \lstinline{AddRoundKey}
            round function  alone,
      \item the  $1$-st to $9$-th rounds consist  of the
            \lstinline{SubBytes}, \lstinline{ShiftRows},  \lstinline{MixColumns} then  \lstinline{AddRoundKey}
            round functions,
            and
      \item the $10$-th           round  consists of the
            \lstinline{SubBytes}, \lstinline{ShiftRows}                          then  \lstinline{AddRoundKey}
            round functions.
      \end{itemize}

\item More concretely, it is an $8$-bit, memory-constrained implementation
      with the S-box held as a $\SI{256}{\byte}$ look-up table in memory. 
      In line with the goal of minimising the memory footprint, the round
      keys are not pre-computed: each encryption takes the cipher key and
      evolves it forward, step-by-step, to form successive round keys for
      use during key addition.  
\item The fault specification is {\em either} a comma-separated line of the
      form
      \[
      r, f, p, i, j
      \]
      where

      \begin{enumerate}
      \item $r$  
            (represented as a decimal integer string)
            specifies the round in which the fault occurs,
            implying $0 \leq r < 11$,
      \item $f$
            (represented as a decimal integer string)
            specifies the round function in which the fault occurs via
            \[
            f = \left\{\begin{array}{ll@{\;}l@{\;}l}
                       0 & \mbox{~for a fault in the} & \mbox{\lstinline{AddRoundKey}} & \mbox{round function} \\
                       1 & \mbox{~for a fault in the} & \mbox{\lstinline{SubBytes}   } & \mbox{round function} \\
                       2 & \mbox{~for a fault in the} & \mbox{\lstinline{ShiftRows}  } & \mbox{round function} \\
                       3 & \mbox{~for a fault in the} & \mbox{\lstinline{MixColumns} } & \mbox{round function}
                       \end{array}
                \right.
            \]
      \item $p$
            (represented as a decimal integer string)
            specifies whether the fault occurs before or after execution 
            of the round function via
            \[
            p = \left\{\begin{array}{ll@{\;}l@{\;}l}
                       0 & \mbox{~for a fault} & \mbox{before} & \mbox{the round function} \\
                       1 & \mbox{~for a fault} & \mbox{after}  & \mbox{the round function} 
                       \end{array}
                \right.
            \]
            and
      \item $i$ and $j$
            (both represented as decimal integer strings),
            specify the row and column of the state matrix which the 
            fault occurs in,
            implying $0 \leq i, j < 4$
      \end{enumerate}

      \noindent
      {\em or} a blank line.  In the latter case, {\em no} fault will 
      be induced; this will obviously yield the correctly encrypted 
      ciphertext for a given plaintext.
\item The fault randomises one element of the state matrix: this can be
      captured by writing the resulting value as $s_{i,j} \XOR \delta$ 
      where $\delta$ is a (random) difference introduced by the fault.
\IfEq{CONF(ALLOW_ZERO,CID)}{1}{%
      Keep in mind that faults st. $\delta = 0$ {\em are} possible, and
      since
      $
      s_{i,j} \XOR \delta = s_{i,j} \XOR 0 = s_{i,j} ,
      $
      this effectively means no fault will be induced in such cases.
}{%
      You can assume $\delta \neq 0$, i.e., a fault will {\em always} be
      induced {\em if} you attempt to do so.
}%
\end{itemize}

% -----------------------------------------------------------------------------

\levelup{Tasks}

\begin{enumerate}
\item Write a program that simulates the adversary $\PARTY{E}$ by attacking
      the simulated target, or, more specifically, that recovers the target 
      material $\PRI{\CHAL{k}}$.  
      When executed using a command of the form
      \[
      \mbox{\lstinline[language={bash}]|./attack \$\{USER\}.D|}
      \]
      the attack should be invoked on the simulated target named (not some
      hard-coded alternative).  Use \lstinline[language={bash}]{stdout} to 
      print 
      a) any intermediate output you deem relevant, followed finally by 
      b) two lines which clearly detail the target material recovered plus
         the total number of interactions with the attack target.
\item Answer the exam-style questions in 
      \lstinline[language={bash}]|${ARCHIVE}/CONF(ARCHIVE_PATH,CID)/${USER}.exam|.
\end{enumerate}

% =============================================================================
