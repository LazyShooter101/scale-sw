% Copyright (C) 2017 Daniel Page <dan@phoo.org>
%
% Use of this source code is restricted per the CC BY-SA license, a copy of
% which can be found via http://creativecommons.org (and should be included 
% as LICENSE.txt within the associated archive or repository).

% =============================================================================

\leveldown{Background}

Imagine you are tasked with attacking a server, denoted $\PARTY{D}$, 
which houses a $64$-bit Intel Core2 processor.  
$\PARTY{D}$ is used by several front-line e-commerce servers, which offload
computation relating to TLS handshakes.  More specifically, $\PARTY{D}$ is 
used to compute an RSA decryption (in software, of the pre-master secret)
whenever RSA-based key exchange is required.
By leveraging access to the network, 
an attacker $\PARTY{E}$ can interact with $\PARTY{D}$ as follows:

\begin{center}
% Copyright (C) 2017 Daniel Page <dan@phoo.org>
%
% Use of this source code is restricted per the CC BY-SA license, a copy of
% which can be found via http://creativecommons.org (and should be included 
% as LICENSE.txt within the associated archive or repository).

\begin{tikzpicture}

% =============================================================================

% configure
 \tikzset{node distance={6cm},party/.style={rectangle,minimum width={2.0cm},minimum height={2.0cm}}}

% adversary and target device
 \node [                party] (E) {$\PARTY{E}$} ;
 \node [right of=E,draw,party] (D) {$\PARTY{D}$} ;

% embedded material
 \node at (D.south east) [anchor=south east,draw,rectangle] {\tiny $\PRI{\CHAL{k}}$} ;

%   direct input and output
 \draw [       >=stealth,->] (E.30)    to              
                             node [above]             {$\PUB{P}$} 
                             (E.30    -| D.west)  ;
 \draw [       >=stealth,<-] (E.330)   to              
                             node [below]             {$\PUB{Q}$} 
                             (E.330   -| D.west)  ;

% indirect input and output
%\draw [dashed,>=stealth,<-] (D.north) to [bend right] 
%                            node [above]             {} 
%                            (D.north -| E.north) ;
%\draw [dashed,>=stealth,->] (D.south) to [bend left]  
%                            node [below]             {} 
%                            (D.south -| E.south) ;

% computation
 \draw [       >=stealth,->] (D.30)    to [bend left,in=90,out=90,looseness=2] 
                             node [right,anchor=west] {$\PRI{Q} = \PMUL{\PRI{\CHAL{k}}}{\PUB{P}}$}
                             (D.330) ;

% =============================================================================

\end{tikzpicture}

\end{center}

\noindent
That is, in each interaction $\PARTY{E}$ can (adaptively) send 
a chosen RSA ciphertext
to $\PARTY{D}$; the device will
decrypt that ciphertext under the fixed, unknown RSA private key $\TUPLE{ \PUB{\CHAL{N}}, \PRI{\CHAL{d}} }$ 
and produce 
the corresponding plaintext.
In addition, $\PARTY{E}$ can measure the time $\PARTY{D}$ takes to execute 
each operation: it approximates this 
(keeping in mind there may be some experimental noise) 
by simply timing how long $\PARTY{D}$ takes to respond with $\PUB{m}$ once
provided with $\PUB{c}$.

% -----------------------------------------------------------------------------

\levelstay{Material}

\leveldown{\lstinline[language={bash}]|\$\{ARCHIVE\}/CONF(ARCHIVE_PATH,CID)/\$\{USER\}.D|}

This executable simulates the attack target $\PARTY{D}$.  When executed it 
reads the following input

\begin{itemize}
\item $\PUB{c}$,
      an RSA ciphertext
      (represented as a                   hexadecimal integer string),
\end{itemize}

\noindent
from \lstinline[language={bash}]{stdin} and writes the following output

\begin{itemize}
\item $\LEAK$,
      an execution time measured in clock cycles
      (represented as a                       decimal integer string),
      and
\item $\PUB{m}$,
      an RSA plaintext
      (represented as a                   hexadecimal integer string),
\end{itemize}

\noindent
to \lstinline[language={bash}]{stdout}, in both cases with one field per 
line.  Execution continues this way, i.e., by repeatedly reading input 
then writing output, until it is forcibly terminated (or crashes).  
Note that:

\begin{itemize}
\item $\PARTY{D}$ houses a $64$-bit processor, so it uses a base-$2^{64}$ 
      representation of multi-precision integers throughout.  Put another 
      way, since $w = 64$, it selects $b = 2^{w} = 2^{64}$.
\item Rather than a CRT-based approach, $\PARTY{D}$ uses a left-to-right 
      binary exponentiation~\cite[Section 2.1]{SCALE:Gordon:85} algorithm
      to compute 
      $
      \PUB{m} = \PUB{c}^{\PRI{\CHAL{d}}} \pmod{\PUB{\CHAL{N}}} .
      $
      Montgomery multiplication~\cite{SCALE:Montgomery:85} is harnessed to
      improve efficiency; following notation in~\cite{SCALE:KocAcaKal:96}, 
      a CIOS-based~\cite[Section 5]{SCALE:KocAcaKal:96} approach is used
      to integrate\footnote{%
      \ALG{MonPro}~\cite[Section 2]{SCALE:KocAcaKal:96} perhaps offers a 
      more direct way to explain this: steps $1$ and $2$ are the integer 
      multiplication and Montgomery reduction written both separately and 
      abstractly, whereas $\PARTY{D}$ realises them concretely using the 
      integrated CIOS algorithm.
      } an integer multiplication with a subsequent Montgomery reduction.
\item Following the notation in~\cite{SCALE:Gordon:85}, $\PRI{\CHAL{d}}$
      is assumed to have $l+1$ bits st. $\PRI{\CHAL{d}_l} = 1$.  However, 
      it has been (artificially) selected st. 
      $
      0 \leq \PRI{\CHAL{d}} < 2^{CONF(LOG_D_MAX,CID)}
      $
      to limit the mount of time an attack requires: you should not rely 
      on this fact, implying your attack should succeed for {\em any}
      $\PRI{\CHAL{d}}$ given enough time.
\end{itemize}

\levelstay{\lstinline[language={bash}]|\$\{ARCHIVE\}/CONF(ARCHIVE_PATH,CID)/\$\{USER\}.conf|}

This file represents a set of attack parameters, with everything (e.g.,
all public values) $\PARTY{E}$ has access to by default.  It contains 

\begin{itemize}
\item $\PUB{\CHAL{N}}$,
      an RSA modulus
      (represented as a                   hexadecimal integer string),
      and
\item $\PUB{\CHAL{e}}$,
      an RSA public exponent
      (represented as a                   hexadecimal integer string),
      st. $\PUB{\CHAL{e}} \cdot \PRI{\CHAL{d}} \equiv 1 \pmod{\PRI{\Phi(\CHAL{N})}}$,
\end{itemize}

\noindent
with one field per line.
More specifically, this represents the RSA public key 
$
\TUPLE{ \PUB{\CHAL{N}}, \PUB{\CHAL{e}} }
$
associated with the unknown RSA private key 
$
\TUPLE{ \PUB{\CHAL{N}}, \PRI{\CHAL{d}} } 
$
embedded in $\PARTY{D}$.

\levelstay{\lstinline[language={bash}]|\$\{ARCHIVE\}/CONF(ARCHIVE_PATH,CID)/\$\{USER\}.R|}

In many side-channel and fault attacks, the attacker is able to perform an 
initial profiling or calibration phase: a common rationale for doing so is 
to support selection or fine-tuning of parameters for a subsequent attack.  
With this in mind, a simulated replica $\PARTY{R}$ of the attack target is 
also provided.
$\PARTY{R}$ is identical to $\PARTY{D}$, bar accepting input of the form

\begin{itemize}
\item $\PUB{\REPL{c}}$,
      an RSA ciphertext
      (represented as a                   hexadecimal integer string),
\item $\PUB{\REPL{N}}$,
      an RSA modulus
      (represented as a                   hexadecimal integer string),
      and
\item $\PUB{\REPL{d}}$,
      an RSA private exponent
      (represented as a                   hexadecimal integer string),
\end{itemize}

\noindent
on \lstinline[language={bash}]{stdin} (again with one field per line):
in contrast to $\PARTY{D}$, this means $\PARTY{R}$ uses the
{\em chosen} RSA private key $\TUPLE{ \PUB{\REPL{N}}, \PUB{\REPL{d}} }$ 
to decrypt the
     chosen  RSA  ciphertext $\PUB{\REPL{c}}$.
Keep in mind that $\PARTY{R}$ uses Montgomery multiplication, so functions
correctly iff. $\gcd( \PUB{\REPL{N}}, b ) = 1$.

% -----------------------------------------------------------------------------

\levelup{Tasks}

\begin{enumerate}
\item Write a program that simulates the adversary $\PARTY{E}$ by attacking
      the simulated target, or, more specifically, that recovers the target 
      material $\PRI{\CHAL{d}}$.  
      When executed using a command of the form
      \[
      \mbox{\lstinline[language={bash}]|./attack \$\{USER\}.D \$\{USER\}.conf|}
      \]
      the attack should be invoked on the simulated target named (not some
      hard-coded alternative).  Use \lstinline[language={bash}]{stdout} to 
      print 
      a) any intermediate output you deem relevant, followed finally by 
      b) two lines which clearly detail the target material recovered plus
         the total number of interactions with the attack target.
\item Answer the exam-style questions in 
      \lstinline[language={bash}]|${ARCHIVE}/CONF(ARCHIVE_PATH,CID)/${USER}.exam|.
\end{enumerate}

% =============================================================================
