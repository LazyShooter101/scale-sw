% Copyright (C) 2017 Daniel Page <dan@phoo.org>
%
% Use of this source code is restricted per the CC BY-SA license, a copy of
% which can be found via http://creativecommons.org (and should be included 
% as LICENSE.txt within the associated archive or repository).

% =============================================================================

\leveldown{Background}

Imagine you are tasked with attacking a device, denoted $\PARTY{D}$, 
which houses an $8$-bit Intel $8051$ processor.
$\PARTY{D}$ represents an ISO/IEC 7816 compliant contact-based smart-card
chip, used within larger devices as a co-processor module: it has support
for secure key generation and storage, plus off-load of some cryptographic 
operations, such as AES, via a standardised protocol (or API).
By leveraging physical access, 
an attacker $\PARTY{E}$ can interact with $\PARTY{D}$ as follows:

\begin{center}
% Copyright (C) 2017 Daniel Page <dan@phoo.org>
%
% Use of this source code is restricted per the CC BY-SA license, a copy of
% which can be found via http://creativecommons.org (and should be included 
% as LICENSE.txt within the associated archive or repository).

\begin{tikzpicture}

% =============================================================================

% configure
 \tikzset{node distance={6cm},party/.style={rectangle,minimum width={2.0cm},minimum height={2.0cm}}}

% adversary and target device
 \node [                party] (E) {$\PARTY{E}$} ;
 \node [right of=E,draw,party] (D) {$\PARTY{D}$} ;

% embedded material
 \node at (D.south east) [anchor=south east,draw,rectangle] {\tiny $\PRI{\CHAL{k}}$} ;

%   direct input and output
 \draw [       >=stealth,->] (E.30)    to              
                             node [above]             {$\PUB{P}$} 
                             (E.30    -| D.west)  ;
 \draw [       >=stealth,<-] (E.330)   to              
                             node [below]             {$\PUB{Q}$} 
                             (E.330   -| D.west)  ;

% indirect input and output
%\draw [dashed,>=stealth,<-] (D.north) to [bend right] 
%                            node [above]             {} 
%                            (D.north -| E.north) ;
%\draw [dashed,>=stealth,->] (D.south) to [bend left]  
%                            node [below]             {} 
%                            (D.south -| E.south) ;

% computation
 \draw [       >=stealth,->] (D.30)    to [bend left,in=90,out=90,looseness=2] 
                             node [right,anchor=west] {$\PRI{Q} = \PMUL{\PRI{\CHAL{k}}}{\PUB{P}}$}
                             (D.330) ;

% =============================================================================

\end{tikzpicture}

\end{center}

\noindent
That is, in each interaction $\PARTY{E}$ can (adaptively) send 
a chosen AES  plaintext
to $\PARTY{D}$; the device will
encrypt that  plaintext under the fixed, unknown AES cipher key $\PRI{\CHAL{k}}$,
and produce 
the corresponding ciphertext.
In addition, $\PARTY{E}$ can measure the time $\PARTY{D}$ takes to execute 
each operation: it approximates this 
(keeping in mind there may be some experimental noise) 
by simply timing how long $\PARTY{D}$ takes to respond with $\PUB{c}$ once
provided with $\PUB{m}$.

% -----------------------------------------------------------------------------

\levelstay{Material}

\leveldown{\lstinline[language={bash}]|\$\{ARCHIVE\}/CONF(ARCHIVE_PATH,CID)/\$\{USER\}.D|}

This executable simulates the attack target $\PARTY{D}$.  When executed it 
 reads the following  input (one field per line)

\begin{itemize}
\item $\PUB{m}$,
      a  ${1}$-block AES plaintext
      (represented as a  length-prefixed, hexadecimal octet   string),
\end{itemize}

\noindent
from \lstinline[language={bash}]{stdin},
and 
writes the following output (one field per line)

\begin{itemize}
\item $\PUB{c}$,
      a  ${1}$-block AES ciphertext
      (represented as a  length-prefixed, hexadecimal octet   string),
\end{itemize}

\noindent
to   \lstinline[language={bash}]{stdout}.
Execution continues this way, i.e., repeatedly reading input then writing 
output, until $\PARTY{D}$ is forcibly terminated (or crashes).
Note that:

\begin{itemize}
\item $\PARTY{D}$ uses an AES-128 implementation,
      which clearly implies $128$-bit block and cipher key lengths;
      following the notation in~\cite[Figure 5]{SCALE:FIPS:197:01}, the 
      fact that 
      $
      Nb =  4
      $ 
      and 
      $
      Nr = 10
      $ 
      means a $( 4 \times 4 )$-element state matrix is used in a total of 
      $11$ rounds.
\item More concretely, it is an $8$-bit, memory-constrained implementation
      with the S-box held as a $\SI{256}{\byte}$ look-up table in memory. 
      In line with the goal of minimising the memory footprint, the round
      keys are not pre-computed: each encryption takes the cipher key and
      evolves it forward, step-by-step, to form successive round keys for
      use during key addition.  

      Crucially, the implementation realises 
      \lstinline[language={C}]|xtime| (i.e., multiplication-by-$\IND{x}$,
      as used in the \lstinline{MixColumns} round function) by computing
      it via

      \begin{lstlisting}[language={C},gobble={6},frame={single},basicstyle={\ttfamily\small}]
      uint8_t xtime( uint8_t a ) {
        if( a & 0x80 ) {
          return 0x1B ^ ( a << 1 );
        }
        else {
          return        ( a << 1 );
        }
      }
      \end{lstlisting}

      \noindent
      rather than, e.g., pre-computed and then realised via accesses into 
      a look-up table.  Note that the resulting computation is {\em not} 
      data-independent.
\end{itemize}

\levelstay{\lstinline[language={bash}]|\$\{ARCHIVE\}/CONF(ARCHIVE_PATH,CID)/\$\{USER\}.R|}

In many side-channel and fault attacks, the attacker is able to perform an 
initial profiling or calibration phase: a common rationale for doing so is 
to support selection or fine-tuning of parameters for a subsequent attack.  
With this in mind, a simulated replica $\PARTY{R}$ of the attack target is 
also provided.
$\PARTY{R}$ is identical to $\PARTY{D}$, bar the fact it
 reads the following  input (one field per line)

\begin{itemize}
\item $\PUB{\REPL{m}}$,
      a  ${1}$-block     AES  plaintext
      (represented as a  length-prefixed, hexadecimal octet   string),
      and            
\item $\PUB{\REPL{k}}$,
      an                 AES cipher key
      (represented as a  length-prefixed, hexadecimal octet   string),
\end{itemize}

\noindent
from \lstinline[language={bash}]{stdin}:
in contrast to $\PARTY{D}$, this means $\PARTY{R}$ uses the
{\em chosen} cipher key $\PUB{\REPL{k}}$ 
to encrypt the
     chosen  plaintext  $\PUB{\REPL{m}}$.

% -----------------------------------------------------------------------------

\levelup{Tasks}

\begin{enumerate}
\item Write a program that simulates the adversary $\PARTY{E}$ by attacking
      the simulated target, or, more specifically, that recovers the target 
      material $\PRI{\CHAL{k}}$.  
      When executed using a command of the form
      \[
      \mbox{\lstinline[language={bash}]|./attack \$\{USER\}.D|}
      \]
      the attack should be invoked on the simulated target named (not some
      hard-coded alternative).  Use \lstinline[language={bash}]{stdout} to 
      print 
      a) any intermediate output you deem relevant, followed finally by 
      b) two lines which clearly detail the target material recovered plus
         the total number of interactions with the attack target.
\item Answer the exam-style questions in 
      \lstinline[language={bash}]|${ARCHIVE}/CONF(ARCHIVE_PATH,CID)/${USER}.exam|.
\end{enumerate}

% =============================================================================
