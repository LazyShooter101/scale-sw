Consider the following Python $3.x$ program

\begin{lstlisting}[language={Python},frame={single},basicstyle={\ttfamily\small}]
import binascii

def octetstr2bytes( x ) :
  t = x.split( ':' ) ; n = int( t[ 0 ], 16 ) ; x = binascii.a2b_hex( t[ 1 ] )

  if ( n != len( x ) ) :
    raise ValueError
  else :
    return x

def bytes2octetstr( x ) :
  n = '{0:02X}'.format( len( x ) ) ; x = binascii.b2a_hex( x ).decode( 'ascii' ).upper()

  return n + ':' + x

def bytes2seq( x ) :
  return            [ int( t ) for t in x ]

def seq2bytes( x ) :
  return bytearray( [ int( t ) for t in x ] )

t_0 = [ 0xDE, 0xAD, 0xBE, 0xEF ]
t_1 = bytes2octetstr( seq2bytes( t_0 ) )

print( 'type( t_0 ) = {0!s:14s} t_0 = {1!s}'.format( type( t_0 ), t_0 ) )
print( 'type( t_1 ) = {0!s:14s} t_1 = {1!s}'.format( type( t_1 ), t_1 ) )

t_2 = '04:DEADBEEF'
t_3 = bytes2seq( octetstr2bytes( t_2 ) )

print( 'type( t_2 ) = {0!s:14s} t_2 = {1!s}'.format( type( t_2 ), t_2 ) )
print( 'type( t_3 ) = {0!s:14s} t_3 = {1!s}'.format( type( t_3 ), t_3 ) )
\end{lstlisting}

\noindent
which, when executed, produces

\begin{lstlisting}[language={bash},  frame={single},basicstyle={\ttfamily\small}] 
type( t_0 ) = <class 'list'>   t_0 = [222, 173, 190, 239]
type( t_1 ) = <class 'str'>    t_1 = 04:DEADBEEF
type( t_2 ) = <class 'str'>    t_2 = 04:DEADBEEF
type( t_3 ) = <class 'list'>   t_3 = [222, 173, 190, 239]
\end{lstlisting}

\noindent
as output.  This is intended to illustrate that

\begin{itemize}
\item If you start with a
      list (or sequence)
      \lstinline[language={Python}]|t_0|, 
      then 
      \lstinline[language={Python}]|seq2bytes| 
      and 
      \lstinline[language={Python}]|bytes2octetstr|
      will convert it first into a byte sequence then an octet string:
      the fact that
      $\mbox{\lstinline[language={Python}]|t_1|} = \mbox{\lstinline[language={Python}]|04:DEADBEEF|}$
      implies 
      $n = \RADIX{04}{16} = \RADIX{04}{10}$
      and
      $x = \LIST{ \RADIX{DE}{16}, \RADIX{AD}{16}, \RADIX{BE}{16}, \RADIX{EF}{16} }$,
      i.e., 
      $n = \mbox{\lstinline[language={Python}]|len( t_0 )|}$
      and
      $x = \mbox{\lstinline[language={Python}]|     t_0  |}$,
      so, for example,
      $x_1 = \mbox{\lstinline[language={Python}]|t_0[ 1 ]|} = \RADIX{AD}{16}$.

\item If you start with an
      octet string  
      \lstinline[language={Python}]|t_2|, 
      then 
      \lstinline[language={Python}]|octetstr2bytes| and \lstinline[language={Python}]|bytes2seq|
      will convert it first into a byte sequence then a list (or sequence):
      the fact that
      $\mbox{\lstinline[language={Python}]|t_2|} = \mbox{\lstinline[language={Python}]|04:DEADBEEF|}$
      implies 
      $n = \RADIX{04}{16} = \RADIX{04}{10}$
      and
      $x = \LIST{ \RADIX{DE}{16}, \RADIX{AD}{16}, \RADIX{BE}{16}, \RADIX{EF}{16} }$,
      i.e., 
      $n = \mbox{\lstinline[language={Python}]|len( t_3 )|}$
      and
      $x = \mbox{\lstinline[language={Python}]|     t_3  |}$,
      so, for example,
      $x_0 = \mbox{\lstinline[language={Python}]|t_3[ 0 ]|} = \RADIX{DE}{16}$.
\end{itemize}

\noindent
Note that the conversions are self-consistent in the sense that
$\mbox{\lstinline[language={Python}]|t_0|} = \mbox{\lstinline[language={Python}]|t_3|}$,
or rather 
\[
\mbox{\lstinline[language={Python}]|t_0|} = \mbox{\lstinline[language={Python}]|bytes2seq( octetstr2bytes( bytes2octetstr( seq2bytes( t_0 ) ) ) )|} .
\]
