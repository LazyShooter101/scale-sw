Consider the following Python program

\begin{lstlisting}[language={Python}]
import binascii

def str2seq( x ) :
  return          [ ord( t ) for t in x ]

def seq2str( x ) :
  return ''.join( [ chr( t ) for t in x ] )

def octetstr2str( x ) :
  t = x.split( ':' ) ; n = int( t[ 0 ], 16 ) ; x = binascii.a2b_hex( t[ 1 ] )

  if( n != len( x ) ) :
    raise ValueError
  else :
    return x

def str2octetstr( x ) :
  return ( '%02X' % ( len( x ) ) ) + ':' + ( binascii.b2a_hex( x ) )

t_0 = '\xDE\xAD\xBE\xEF' ; t_1 = [ 0xDE, 0xAD, 0xBE, 0xEF ]

t_2 = str2octetstr(          t_0   )
t_3 = str2octetstr( seq2str( t_1 ) )

print "type( t_0 ) = %-18s t_0 = %s" % ( type( t_0 ), repr( t_0 ) )
print "type( t_1 ) = %-18s t_1 = %s" % ( type( t_1 ), repr( t_1 ) )
print "type( t_2 ) = %-18s t_2 = %s" % ( type( t_2 ), repr( t_2 ) )
print "type( t_3 ) = %-18s t_3 = %s" % ( type( t_3 ), repr( t_3 ) )

t_4 = '04:8BADF00D'

t_5 =          octetstr2str( t_4 )
t_6 = str2seq( octetstr2str( t_4 ) )

print "type( t_4 ) = %-18s t_4 = %s" % ( type( t_4 ), repr( t_4 ) )
print "type( t_5 ) = %-18s t_5 = %s" % ( type( t_5 ), repr( t_5 ) )
print "type( t_6 ) = %-18s t_6 = %s" % ( type( t_6 ), repr( t_6 ) )
\end{lstlisting}

\noindent
which, when executed, produces

\begin{lstlisting}[language={bash}]
type( t_0 ) = <type 'str'>       t_0 = '\xde\xad\xbe\xef'
type( t_1 ) = <type 'list'>      t_1 = [222, 173, 190, 239]
type( t_2 ) = <type 'str'>       t_2 = '04:deadbeef'
type( t_3 ) = <type 'str'>       t_3 = '04:deadbeef'
type( t_4 ) = <type 'str'>       t_4 = '04:8BADF00D'
type( t_5 ) = <type 'str'>       t_5 = '\x8b\xad\xf0\r'
type( t_6 ) = <type 'list'>      t_6 = [139, 173, 240, 13]
\end{lstlisting}

\noindent
as output.  This is intended to illustrate that

\begin{itemize}

\item If you start with 
      byte string (or array)
      \lstinline[language={Python}]|t_0|, 
      then 
                             \lstinline[language={Python}]|str2octetstr|
      convert this into the 
      octet string,
      you expect that
      $\mbox{\lstinline[language={Python}]|t_2|} = \mbox{\lstinline[language={Python}]|04:DEADBEEF|}$
      implies 
      $n = \RADIX{04}{16} = \RADIX{04}{10}$
      and
      $x = \LIST{ \RADIX{DE}{16}, \RADIX{AD}{16}, \RADIX{BE}{16}, \RADIX{EF}{16} }$
      st.
      $\mbox{\lstinline[language={Python}]|t_0[ 0 ]|} = \RADIX{DE}{16} = x_0$.

\item If you start with 
      byte sequence 
      \lstinline[language={Python}]|t_1|, 
      then 
      \lstinline[language={Python}]|seq2str| and \lstinline[language={Python}]|str2octetstr|
      convert this into the 
      octet string,
      you expect that
      $\mbox{\lstinline[language={Python}]|t_3|} = \mbox{\lstinline[language={Python}]|04:DEADBEEF|}$
      implies 
      $n = \RADIX{04}{16} = \RADIX{04}{10}$
      and
      $x = \LIST{ \RADIX{DE}{16}, \RADIX{AD}{16}, \RADIX{BE}{16}, \RADIX{EF}{16} }$
      st.
      $\mbox{\lstinline[language={Python}]|t_1[ 1 ]|} = \RADIX{AD}{16} = x_1$.

\item If you start with 
      octet string  
      \lstinline[language={Python}]|t_4|, 
      then 
      \lstinline[language={Python}]|octetstr2str|
      convert this into the byte array,    
      you expect that
      $\mbox{\lstinline[language={Python}]|t_4|} = \mbox{\lstinline[language={Python}]|04:8BADF00D|}$
      implies 
      $n = \RADIX{04}{16} = \RADIX{04}{10}$
      and
      $x = \LIST{ \RADIX{8B}{16}, \RADIX{AD}{16}, \RADIX{F0}{16}, \RADIX{0D}{16} }$
      st.
      $\mbox{\lstinline[language={Python}]|t_5[ 2 ]|} = \RADIX{F0}{16} = x_2$.

\item If you start with  
      octet string  
      \lstinline[language={Python}]|t_4|, 
      then 
      \lstinline[language={Python}]|octetstr2str| and \lstinline[language={Python}]|str2seq|
      convert this into the byte sequence, 
      you expect that
      $\mbox{\lstinline[language={Python}]|t_4|} = \mbox{\lstinline[language={Python}]|04:8BADF00D|}$
      implies 
      $n = \RADIX{04}{16} = \RADIX{04}{10}$
      and
      $x = \LIST{ \RADIX{8B}{16}, \RADIX{AD}{16}, \RADIX{F0}{16}, \RADIX{0D}{16} }$
      st.
      $\mbox{\lstinline[language={Python}]|t_6[ 3 ]|} = \RADIX{0D}{16} = x_3$.

\end{itemize}
