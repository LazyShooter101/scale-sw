The term octet\footnote{
\url{http://en.wikipedia.org/wiki/Octet_(computing)}
} is normally used as a synonym for byte, most often within the context
of communication (and computer networks).  Using octet is arguably more 
precise than byte, in that the former is {\em always} $8$ bits whereas
the latter {\em can}\footnote{
\url{http://en.wikipedia.org/wiki/Byte}, for example, details the fact
that the term ``byte'' can be and has been interpreted to mean
a) a group of $n$ bits for $n < w$ (i.e., smaller than the word size),
b) the data type used to represent characters,
   or
c) the (smallest) unit of addressable data in memory:
although POSIX mandates $8$-bit bytes, for example, each of these cases
permits an alternative definition.  
} differ.  A string is a sequence of characters, and so, by analogy, an
octet string\footnote{
Note the octet string terminology stems from ASN.1 encoding; see, e.g., 
\url{http://en.wikipedia.org/wiki/Abstract_Syntax_Notation_One}.
} is a sequence of octets: ignoring some corner cases, it is reasonable 
to use the term ``octet string'' as a synonym for ``byte sequence''.

To represent a given byte sequence, we use what can be formally termed
a (little-endian) length-prefixed, hexadecimal octet string.  However,
doing so requires some explanation: each element of that term relates
to a property of the representation, where we define
a) little-endian\footnote{
   \url{http://en.wikipedia.org/wiki/Endianness}
   } to mean, if read left-to-right,
   the first octet represents the     $0$-th element of the source byte sequence
   and
   the last  octet represents the $(n-1)$-st element of the source byte sequence,
b) length-prefixed\footnote{
   \url{http://en.wikipedia.org/wiki/String_(computer_science)}
   } to mean $n$, the length of the source byte sequence, is prepended 
   to the octet string as a single $8$-bit\footnote{
   Although it simplifies the challenge associated with parsing such a 
   representation, note that use of an $8$-bit length implies an upper 
   limit of $255$ elements in the associated byte sequence.
   } length or ``header'' octet,
   and
c) hexadecimal\footnote{
   \url{http://en.wikipedia.org/wiki/Hexadecimal}
   } to mean each octet is represented by using $2$ hexadecimal digits.  
   Note that, confusingly, hexadecimal digits within each pair will be
   big-endian: if read left-to-right, the most-significant is first.
For convenience, we assume the term octet string is a catch-all implying
all such properties from here on.

An example likely makes all of the above {\em much} clearer: certainly
there is nothing complex involved.  Concretely, consider a $16$-element 
byte sequence 
\[
\mbox{\lstinline[language={C}]|uint8_t x[ 16 ] = \{ 0, 1, 2, 3, 4, 5, 6, 7, 8, 9, 10, 11, 12, 13, 14, 15 \}|}
\]
defined using C.  This would be represented as
\[
\REP{x} ~=~ \mbox{\lstinline|10:000102030405060708090A0B0C0D0E0F|}
\]
using a colon to separate the length and value fields:

\begin{itemize}
\item the length (LHS of the colon) is the integer
      $
      n = \RADIX{10}{16} = \RADIX{16}{10} ,
      $
      and
\item the value  (RHS of the colon) is the byte sequence
      $
      x = \LIST{ \RADIX{00}{16}, \RADIX{01}{16}, \ldots, \RADIX{0F}{16} } = \LIST{ \RADIX{0}{10}, \RADIX{1}{10}, \ldots, \RADIX{15}{10} } \equiv \mbox{\lstinline[language={C}]|x|} .
      $
\end{itemize}
